Working together with multiple people has the requirement, that these people are able to communicate with some kind of communication type.
In cooperative computer games, players are able to communicate using verbal and physical non-verbal communication types. However, computer games additionally have the possibility to use virtual types of non-verbal communication, which sometimes cannot be done in the real world.
This bachelor thesis describes communication types that can be used within computer games, as well as already existing game mechanics and their classification.
To conduct a study, a prototype was built. This prototype is a network based cooperative multiplayer puzzle game, which can be played by two players in real time. For implemented communication mechanics, players are able to use four different types of communication tools: different kinds of pings, various types of emojis, permanent placeable decals, as well as the recording of the player character position, which will be played in a loop after the recording has finished.
During an online based user study, attendees have 30 minutes to pass all nine levels. 
Verbal communication is not allowed during the testing. For communication, players are only able to use the given communication mechanics.
The aim of this thesis is to investigate, how players are using the provided cooperative communication mechanics, as well as how game communication mechanics can be implemented in a self made computer game. 