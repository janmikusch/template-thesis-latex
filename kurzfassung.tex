Die Zusammenarbeit mehrerer Leute setzt vorraus, das man über irgend eine Art und Weiße miteinander kommunizieren kann. In kooperativen Computerspielen bestehen neben verbalen und physischen non-verbalen Kommunikationsmitteln zusätzlich virtuelle non-verbale Kommunikationsmittel.
Diese Bachelorarbeit beschreibt sowohl Kommunikationsformen, die in Computerspielen eingesetzt werden, als auch bereits existierende virtuelle Kommunikationsmechaniken, sowie eine Einteilung dieser kooperativen Mechaniken. 
Für die Durchführung einer Studie wurde ein Prototyp gebaut. Dieser Prototyp ist ein netzwerkbasiertes, kooperatives Multiplayer Puzzle-Spiel, welches von zwei Spielern bzw. Spielerinnen in Echtzeit gespielt werden kann. Als im Spiel implementierte Kommunikationsmechaniken dienen vier verschiedene kooperative Mechaniken: verschiedene Arten von Pings, verschiedene Emojis, das Anbringen von permanenten Bodenzeichnungen sowie die Aufzeichnung der Spielerposition, die in einer Schleife nach der Aufnahme abgespielt wird. 
Neben der Spielidee wird auch die Implementierung der Mechaniken beschrieben.
Während einer online-basierten User-Studie haben die Teilnehmer und Teilnehmerinnen 30 Minuten Zeit, alle neun Level zu absolvieren. Dabei ist verbale Kommunikation nicht erlaubt, wodurch die Teilnehmer und Teilnehmerinnen auf die vier implementierten kooperativen Kommunikationsmechaniken angewiesen sind. Ziel dieser Arbeit ist es zu untersuchen, wie die Teilnehmer und Teilnehmerinnen die Mechaniken benutzen, sowie Kommunikationsformen in einem selbstgebauten Prototypen implementiert werden können.