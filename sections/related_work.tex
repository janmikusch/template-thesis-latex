There are several studies about performance improvement through cooperative communication.
\textcite{Vaddi2016Investigating2}[45] showed in their study, that communication has a massive impact on the performance of players. Their participants were assigned four different communication conditions. Having the possibility to talk, using in-game mechanics, both and neither of these possibilities.
Results indicated, that the combination of voice and in-game mechanics occurs to be the highest performing condition.

\textcite{Leavitt2016PingGames}[4346] showed in their paper, that virtual team members employ non-verbal communication strategies to improve their performance up to a point for certain crucial actions. But using this mechanics too much, showed a negative impact as the communication actions negatively, resulting in interruption of gameplay actions.
Additionally, \textcite{Wuertz2017Why2}[1978] looked into, why and how people use gesturing tools in a \textit{Multiplayer Online Battle Arena} game (MOBA), and what degree they view them as important for coordinating with teammates to win matches.

\textcite{Cheung2012CommunicationGaming} wrote about the communication channels and awareness cues in collocated collaborative time-critical gaming. They analysed, how players communicate when they need to be highly focused, but have only analysed the collaboration in an collocated setting.
Also \textcite{Williams2007CanCommunity} reports the influence of voice communication into existing online communities.

Regarding the theoretical part of communication, the work of \textcite{Benford2001CollaborativeEnvironments} and \textcite{Maher2011DesignersEnvironments} provided information about collaborative virtual environments.
Also the work of \textcite{Galantucci2012TheHumans} about embodiment within games and the paper about common ground of \textcite{Clark2004GroundingCommunication.} played an important part of theoretical research.