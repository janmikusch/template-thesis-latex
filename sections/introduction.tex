Cooperation is an anchor of our society and a key ability of people who are working, studying and carrying out most leisure activities together to achieve shared goals \autocite{Morschheuser2017HowGame}[169]. A key part of cooperation is communication. To work together, a team has to share information to work towards a common goal.
Cooperative multiplayer computer games have, besides some exceptions, always a goal, a team is working to achieve, which can be to accomplish a high score, complete a level, win against an enemy team and so on.

The aim of this thesis is to show, how players will interact within a cooperative puzzle game without the possibility to communicate using their voice, in an unfamiliar environment and how non-verbal communication forms can be implemented. 
To provide an unfamiliar game players have never played before, a self-made cooperative puzzle game will be created. This will implement non-verbal communication mechanics, enabling players to communicate during play.

The first section will show, how communication is used in computer games, as well what is necessary to share information successfully. Verbal and non-verbal communication forms will be described and how common ground influences communication, as well as the meaning of embodiment in games.
Afterwards, the influence and reason of game mechanics for in-game communication will be discussed. Additionally, the framework of \textcite{Toups2014ATheory}, which divides cooperative game mechanics into six different types, will be described.

In the implementation part, the game idea as well as all game elements will be explained. 
The implementation of four cooperative game mechanic types will be explained as well. Besides an attention-focusing ping mechanic, an expressive mechanic in form of emojis, environment-modifying decals and the recording of the player position were implemented.
After the theoretical part of the implementation is explained, the following sections will describe, how each game element and all communication mechanics were implemented, using Unity as a game engine, together with Photon as a network-multiplayer framework.

The procedure of the study, followed by the study results, will show how players were using the provided communication mechanics. The whole study was done online, without physical contact to and between the participants.