This study analyses how non-verbal communication mechanics can be implemented in a small cooperative puzzle game. Furthermore, the study showes how these implemented mechanics are used by the participants.

With this study, no causal relation between the duration or the type of relationship may be found. 
A study with a higher amount of participants may show better results regarding this.
Also, a higher variety of experienced players may show more different usages of the implemented mechanics, as well as show how often these are needed. 
As every player had a lot of experience, inexperienced players may handle the communication differently, therefore a study with more participants having different skill-levels would be a topic for the future.

However, the study proves, that it is possible to communicate in new virtual worlds, without any agreement concerning the usage of how the tools are used. Nevertheless, it also shows that a lot of misinterpretation can occur during the communication when missing common ground or not having the possibility to build up a common ground.
\textcite{Vaddi2016Investigating2}[46] state, that the performance of teams using a combination of voice and mechanics is significantly faster, which can be related to having the possibility to negotiate tactics and usage of tools and therefore build their common ground.

Besides the misinterpretation, three teams also stated that they were missing the possibility to show, that they had no clue what to do, and what the other player meant. For that reason, they would like to have a question mark or thinking-emoji to signal their partner about their concerns.
Another aspect three teams mentioned, are the possibility to cancel interactions. Once a button is clicked, there is no way to cancel the interaction, whereby unwanted communication actions influence the study data with unwanted and unnecessary interactions. They also stated, that not having a possibility to cancel the mechanic, was annoying to them, especially when they confused the key-binding to the wanted tool.

When these methods are implemented in a real game, it will also have to be considered, which different options players should have. Some options of emojis and decals were simply not used, as well as the fact that the pull \& pushed ping could have been combined into a single option. As too many choices can slow down the search for a specific option, as stated by two participants, having more play-tests to find the optimal amount of choices is highly recommended.